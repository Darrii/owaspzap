\documentclass[conference]{IEEEtran}
\usepackage{amsmath} % For mathematical formulas
\usepackage{booktabs} % For professional tables
\usepackage{graphicx} % For including figures

\begin{document}

\title{Graph-Based Probabilistic Approach for Automated Vulnerability Chain Detection in Web Security Scanning}

\author{\IEEEauthorblockN{Dariga Yermakhankyzy, Syed Imran Moazzam Shah}
\IEEEauthorblockA{School of Information Technology and Engineering\\ Kazakh-British Technical University\\
Almaty, Kazakhstan\\
Tel.: +7 747 536 2003\\
Email: d\_yermakhankyzy@kbtu.kz }
}

\maketitle

\begin{abstract}
Web application vulnerability scanners identify security weaknesses in isolation. They fail to detect compound attack scenarios where multiple vulnerabilities combine to enable critical breaches. This paper presents a graph-based probabilistic approach for automated detection of multi-stage vulnerability chains in web applications. Our system models vulnerabilities as nodes in a directed graph with probabilistic rules creating edges that represent how attacks can transition between vulnerabilities. A depth-first search algorithm systematically explores the graph to identify chains of length 2-4 while intelligent filtering reduces raw candidates to actionable attack patterns. The system integrates seamlessly with OWASP ZAP through REST API enabling real-time chain analysis within existing security workflows. We evaluated the system on three deliberately vulnerable applications (DVWA Juice Shop WebGoat) containing 842 vulnerabilities. The system identified 19 distinct attack patterns representing realistic compound threats including XSS-enabled CSRF bypasses and configuration weaknesses facilitating SQL injection. Analysis completed in under 16 minutes demonstrating practical applicability for continuous security testing in production environments. Key contributions include a probabilistic rule engine with 53 domain-specific attack patterns encoding real-world exploitation sequences a graph-based detection algorithm with intelligent filtering and normalized risk scoring that helps security teams prioritize remediation of compound threats rather than treating all vulnerabilities equally.
\end{abstract}

\begin{IEEEkeywords}
Web application security, vulnerability chain detection, graph-based analysis, probabilistic modeling, OWASP ZAP, attack path analysis, security testing automation, compound vulnerabilities
\end{IEEEkeywords}

\IEEEpeerreviewmaketitle

\section{Introduction}

Web applications have become critical infrastructure for modern organizations. They handle sensitive data and business-critical operations across e-commerce healthcare finance and government sectors. As these applications grow in complexity they inevitably introduce numerous security vulnerabilities that threaten data integrity confidentiality and availability. Consider this: recent studies show that approximately 3.73\% of tested applications exhibit at least one instance of broken access control vulnerabilities alone \cite{OWASP2025}. This is concerning. Despite substantial organizational investments in security testing detecting and remediating web application vulnerabilities remains challenging due to the sophisticated nature of modern cyber attacks.

Current web application security testing relies heavily on automated vulnerability scanners such as OWASP ZAP \cite{OWASPZAP} Burp Suite and Nikto. These tools employ static analysis dynamic analysis and fuzzing techniques to identify security weaknesses in web applications \cite{Makino2015, Srinivasan2017}. Comparative studies have demonstrated that modern vulnerability scanners can effectively detect individual vulnerabilities including SQL injection cross-site scripting (XSS) and authentication flaws \cite{Potti2025, Mohaidat2024}. However these scanners operate under a fundamental limitation: they identify vulnerabilities in isolation treating each security issue as an independent finding without considering potential interactions or dependencies between multiple vulnerabilities.

This isolated detection approach fails to capture a critical threat vector in web security: multi-stage attack chains. Real-world attackers rarely exploit a single vulnerability to compromise a system. Instead, they craft sophisticated attacks that combine multiple vulnerabilities in sequence. Each vulnerability in the chain enables exploitation of the next, ultimately leading to critical security breaches that individual vulnerability assessments cannot predict \cite{Kasturi2024Priority}.

Consider a practical example: an attacker might first leverage information disclosure to obtain session tokens. Next, they exploit XSS to bypass CSRF protection. Finally, they achieve administrative privilege escalation. This three-stage attack chain appears as three isolated medium-severity findings in traditional scan reports—nothing alarming at first glance. The problem? Security analysts must manually review hundreds of vulnerability findings and mentally correlate them to identify such compound threats. This process is time-consuming, error-prone, and requires significant expertise \cite{Kumar2017}.

Researchers have explored various approaches to address vulnerability correlation and attack path analysis. Statistical correlation methods identify relationships between vulnerabilities based on temporal and spatial patterns \cite{Kasturi2023Understanding}. Attack tree modeling provides theoretical frameworks for representing multi-stage attacks \cite{Mauw2006}. Recent advances in automated penetration testing leverage machine learning and reinforcement learning techniques \cite{Li2024DynPen, Deng2024Pentest}.

Yet these approaches share common limitations. Statistical methods require offline manual analysis and lack real-time processing capabilities. Attack tree frameworks remain primarily theoretical without automated implementation for web applications. ML-based penetration testing tools focus on network-level or IoT scenarios rather than web-specific vulnerability chains \cite{Rak2022}. Perhaps most critically, none of these solutions integrate seamlessly with widely-adopted vulnerability scanners, creating a gap between vulnerability detection and chain analysis.

This paper presents a graph-based probabilistic approach for automated detection of multi-stage attack chains in web applications, integrated with OWASP ZAP. Our system addresses existing limitations through four key contributions: a probabilistic rule engine with 53 domain-specific chain rules encoding real-world attack patterns with confidence scores and bounded boosting (max 2.5× multiplier); a graph-based detection algorithm using depth-first search to identify vulnerability chains of length 2-4 with sub-second performance; an intelligent multi-stage filtering mechanism that reduces tens of thousands of potential chains to actionable findings through on-the-fly deduplication, subchain removal, and pattern-based collapsing (48.6\% reduction from 37 chains to 19 unique patterns); and real-time integration with OWASP ZAP via REST API, enabling automated chain analysis immediately upon scan completion. In our experimental evaluation, the system analyzed 129 vulnerabilities and identified 44 critical attack chains in 0.4 seconds, reducing 37,000 raw chain candidates to 37 unique chains, then to 19 distinct attack patterns through pattern deduplication.

\section{Literature Review}

The detection of security vulnerabilities in web applications has been extensively studied from multiple perspectives. We organize related work into four primary research areas: web vulnerability scanners and testing methodologies, vulnerability correlation and prioritization approaches, attack modeling and chain detection techniques, and machine learning applications in security analysis.

\subsection{Web Vulnerability Scanners and Testing Methodologies}

Automated vulnerability scanners constitute the foundation of modern web application security testing. OWASP ZAP (Zed Attack Proxy) represents one of the most widely adopted open-source security testing tools offering both passive and active scanning capabilities \cite{OWASPZAP}. Recent benchmarking studies have evaluated ZAP's effectiveness in detecting OWASP Top 10 vulnerabilities demonstrating its ability to identify SQL injection XSS and security misconfigurations with varying degrees of accuracy \cite{Potti2025}. Comparative analyses of vulnerability scanners including ZAP Burp Suite Arachni and Nikto have revealed that while each tool exhibits unique strengths all share a common limitation: they detect vulnerabilities independently without correlating findings \cite{Makino2015, Mohaidat2024, Srinivasan2017}.

Automated testing methodologies have evolved to address specific vulnerability classes. Combinatorial testing approaches have been proposed for SQL injection detection systematically generating attack vectors based on input grammars \cite{Bozic2019}. Studies comparing automated versus manual penetration testing have demonstrated that while automation improves efficiency manual analysis remains necessary for identifying complex attack scenarios \cite{Khan2020}. Recent advances in automated penetration testing include expert systems that guide security assessments through threat intelligence \cite{Granata2024} and LLM-enhanced tools that leverage large language models for test case generation \cite{Deng2024Pentest}. However these approaches focus on individual vulnerability detection rather than identifying relationships between multiple security weaknesses. Furthermore existing scanners lack context understanding between vulnerabilities treating each finding as an isolated security issue without considering how multiple flaws might be chained together in a compound attack.

\subsection{Vulnerability Correlation and Prioritization}

The challenge of correlating vulnerability data from multiple sources has gained increasing attention in application security research. Statistical correlation methods have been developed to analyze relationships between vulnerability detection results and real-world attack patterns observed in Web Application Firewall (WAF) logs \cite{Kasturi2023Understanding}. These approaches employ time-series analysis to identify correlations between vulnerability types and actual exploitation attempts providing insights into which combinations of vulnerabilities represent elevated risk \cite{Kasturi2024Priority}.

Attack path prediction represents an advanced application of vulnerability correlation. Recent work has proposed using correlation analysis combined with attack tree modeling and multi-layer perceptrons to predict likely attack sequences based on vulnerability distributions across application layers \cite{Kasturi2024Predicting}. This research demonstrates that vulnerabilities can be mapped to attack trees representing threat models enabling simulation of potential attack paths. However these correlation approaches require manual analysis and offline processing lacking the real-time processing capability necessary for integration into security scanning workflows. Additionally while statistical correlation can identify historical patterns it does not incorporate domain knowledge about vulnerability exploitability and real-world attack probabilities.

Application vulnerability correlation (AVC) tools have emerged in the commercial security space to aggregate and normalize findings from multiple security testing tools \cite{Kasturi2024Priority}. These solutions address the problem of duplicate vulnerabilities reported by different scanners and attempt to prioritize remediation efforts. However they focus primarily on deduplication and risk scoring of individual vulnerabilities rather than identifying multi-stage attack chains.

\subsection{Attack Modeling and Chain Detection}

Theoretical frameworks for modeling attacks have been established through attack tree and attack graph methodologies. Attack trees provide a formal notation for representing how attackers might achieve specific goals through combinations of actions \cite{Mauw2006}. This hierarchical representation captures attack sequences and alternative paths enabling security analysts to reason about system vulnerabilities systematically. However attack tree construction typically requires manual effort and domain expertise limiting their application in automated security testing.

Threat modeling approaches have been developed for specific domains particularly IoT and edge computing environments \cite{Ficco2021}. Expert systems for automated threat modeling and penetration testing have shown promise in IoT ecosystems combining threat intelligence with automated testing frameworks \cite{Rak2022}. These systems demonstrate the feasibility of automating security assessment through rule-based approaches. Nevertheless they are focused on network-level and IoT-specific threats rather than web application vulnerability chains.

Recent advances in automated penetration testing have explored reinforcement learning and deep learning techniques. Deep reinforcement learning has been applied to automated penetration testing in dynamic network scenarios learning optimal attack paths through interaction with target systems \cite{Li2024DynPen}. LLM-empowered penetration testing tools have demonstrated the potential of large language models to automate complex security testing workflows \cite{Deng2023GPT}. While these approaches show promising results in network penetration testing they are not specialized for web application chains and require substantial computational resources for training and execution.

\subsection{Machine Learning and Deep Learning for Vulnerability Detection}

Machine learning techniques have been increasingly applied to vulnerability detection and security analysis. Deep learning-based systems have been developed for source code vulnerability detection employing neural networks to identify security flaws in software \cite{Li2018Vul, Li2022SySeVR}. Systematic literature reviews have examined the application of deep learning to web application security finding that convolutional neural networks long short-term memory networks and deep feedforward networks are commonly used for vulnerability detection \cite{Alaoui2022Deep}.

Specific vulnerability classes have been targeted with machine learning approaches. LSTM encoder-decoder architectures with word embeddings have been proposed for XSS attack detection \cite{Alaoui2023XSS}. These models learn patterns from training data to classify inputs as benign or malicious. However machine learning approaches to web security face significant limitations: they are trained on individual vulnerability patterns rather than attack sequences require substantial labeled training data and produce results that lack interpretability for security practitioners \cite{Chughtai2024}. Furthermore ML-based vulnerability detection focuses on identifying instances of known vulnerability types rather than discovering relationships between multiple vulnerabilities that could form attack chains.

\subsection{Research Gaps}

Despite extensive research in web application security testing several critical gaps remain unaddressed. First existing vulnerability scanners detect security issues in isolation without identifying multi-stage attack chains that represent compound threats. Second vulnerability correlation approaches require manual analysis and offline processing lacking integration with real-time scanning tools. Third attack modeling frameworks remain primarily theoretical or focused on non-web domains such as IoT and network security. Fourth machine learning approaches target individual vulnerability detection rather than attack sequence identification. Our work addresses these gaps by introducing a probabilistic graph-based system that automatically detects vulnerability chains through real-time integration with OWASP ZAP combining domain knowledge encoded in probabilistic rules with efficient graph traversal algorithms.

\section{Methods}

\subsection{Research Gaps}

Contemporary web application security assessment faces four fundamental challenges that limit vulnerability detection and remediation effectiveness. Understanding these gaps is essential for appreciating why automated chain detection represents a necessary advancement.

\textbf{Gap 1: Isolated Vulnerability Detection.} Existing vulnerability scanners including OWASP ZAP, Burp Suite, and Nikto identify individual security weaknesses without analyzing relationships between findings \cite{Makino2015, Mohaidat2024}. When a scanner reports 100-200 vulnerabilities in a typical assessment, security analysts must manually review each finding independently.

Why is this problematic? This approach misses compound attacks where multiple vulnerabilities combine to enable critical breaches. For example, an information disclosure vulnerability leaking authentication tokens becomes significantly more dangerous when combined with a CSRF bypass. Yet scanners report these as separate medium-severity issues rather than identifying the critical attack chain they form.

\textbf{Gap 2: Manual Correlation Analysis.} Security analysts currently perform vulnerability correlation manually, examining hundreds of findings to identify potential attack paths \cite{Kumar2017}. This manual approach proves time-consuming—requiring hours or days for comprehensive analysis of large scan reports. It is also error-prone, as analysts may overlook subtle relationships between vulnerabilities. The consequences are serious: without automated correlation, critical multi-stage attack chains may remain undetected until exploited by attackers who have the time and expertise to identify compound threats.

\textbf{Gap 3: Absence of Probability Assessment.} While statistical correlation methods have been proposed \cite{Kasturi2023Understanding}, they fail to incorporate domain knowledge about vulnerability exploitability and real-world attack probabilities. Existing approaches treat all vulnerability combinations equally. But should they? Certain chains (e.g., XSS leading to CSRF bypass) are significantly more likely to be exploitable than others. This oversight results in either overwhelming analysts with false positive chain candidates or missing genuine attack paths due to overly conservative filtering.

\textbf{Gap 4: Integration Challenges.} Most research tools for vulnerability correlation operate as standalone systems disconnected from the scanning workflow \cite{Kasturi2024Priority}. OWASP ZAP, despite being one of the most popular open-source security scanners, lacks built-in chain detection. This gap forces security teams to export scan results, manually process them through separate correlation tools, and reconcile findings across multiple systems—a workflow that introduces delays and reduces the likelihood of systematic chain analysis.

\subsection{Proposed Approach}

We propose a graph-based probabilistic approach that addresses these gaps through automated, real-time vulnerability chain detection integrated with OWASP ZAP. Our system models vulnerabilities and their relationships as a directed graph, where nodes represent security findings and edges encode probabilistic rules defining how one vulnerability can enable exploitation of another.

The approach consists of four interconnected components. A \textbf{graph representation} transforms vulnerability scan results into a structured format where each vulnerability becomes a node annotated with severity, exploitability, and context information. A \textbf{probabilistic rule engine} encodes 53 domain-specific chain rules capturing real-world attack patterns, such as "XSS enables CSRF bypass with 85\% probability" or "SQL injection leads to privilege escalation with 90\% probability." Each rule specifies source and target vulnerability types, transition probability, and contextual constraints for chain viability.

A \textbf{depth-first search algorithm} traverses the vulnerability graph to identify potential attack chains of length 2-4, applying probability thresholds to filter unlikely paths. The algorithm employs path pruning to avoid exploring chains with cumulative probability below a configurable threshold, maintaining computational efficiency for large vulnerability sets. A \textbf{smart filtering mechanism} processes raw chain candidates to eliminate duplicates and remove subchains—if chain A$\rightarrow$B$\rightarrow$C is detected, the system removes the shorter chain A$\rightarrow$B since it represents a subset of the longer path.

The system integrates with OWASP ZAP through its REST API, enabling automated chain analysis upon scan completion. When a ZAP scan finishes, our system retrieves vulnerability findings, constructs the graph, executes chain detection, and generates an enriched report highlighting critical attack chains alongside individual vulnerabilities. This integration eliminates the workflow gap between vulnerability detection and chain analysis, ensuring compound threats are identified as part of standard security assessment.

\subsection{System Architecture}

Figure 1 presents the complete system architecture, illustrating the eight-stage pipeline from vulnerability scanning to chain detection and reporting. The architecture follows a modular design where each component performs a specific transformation on the data flow.

\begin{figure*}[htbp]
\centering
\includegraphics[width=0.45\textwidth]{experiments/diagrams/figure_1_architecture.png}
\caption{System Architecture: Eight-stage vulnerability chain detection pipeline showing data flow from OWASP ZAP scanning through graph construction, rule application, DFS detection, and intelligent filtering to final report generation.}
\label{fig:architecture}
\end{figure*}

The pipeline begins with OWASP ZAP scanning the target web application using spider-based crawling and active payload injection. The scanner produces a JSON report containing vulnerability alerts with metadata including severity, CWE classification, and affected URLs. Our ZAP Alert Parser processes this report, extracting vulnerability instances and performing signature-based deduplication to eliminate redundant findings (e.g., the same XSS vulnerability detected on multiple similar endpoints).

The Graph Builder constructs a NetworkX directed graph where each unique vulnerability becomes a node. The Probabilistic Rule Engine then applies 53 domain-specific chain rules to create edges between nodes. Each rule encodes attack patterns such as "XSS enables CSRF bypass" with an associated probability reflecting the likelihood that an attacker could exploit the first vulnerability to leverage the second. The rule engine incorporates critical optimizations: precompiled regular expressions for URL pattern matching (10-100$\times$ speedup over on-demand compilation) taxonomy-based LRU caching for vulnerability classification (95\% cache hit rate 1291$\times$ speedup) and bounded boosting with a configurable multiplier cap (default 2.5$\times$ range 1.0-5.0) to prevent probability overflow while allowing context-aware adjustments based on attack surface characteristics (zero-click +30\% data exfiltration +25\% automated exploitation +15\% account takeover +10\%).

The DFS Chain Detector performs depth-first search traversal of the vulnerability graph to identify all potential attack chains of length 2-4. The algorithm applies configurable constraints including minimum chain probability (default 0.65), maximum chains per source node (500), and maximum unique patterns globally (100) to balance comprehensive detection with computational efficiency. During traversal, the system performs on-the-fly deduplication using hash-based signatures, reducing tens of thousands of raw chain candidates to unique attack patterns.

The Smart Filter component applies three reduction stages. First it performs final deduplication to ensure each unique vulnerability sequence appears only once retaining the instance with the highest risk score when duplicates exist. Second it removes subchains—if a longer chain A$\rightarrow$B$\rightarrow$C is detected the system eliminates shorter chains A$\rightarrow$B and B$\rightarrow$C since they represent subsets of the longer attack path. Third pattern-based deduplication collapses chains sharing identical vulnerability type sequences using frozenset-based signatures reducing 37 unique chains to 19 distinct attack patterns (48.6\% reduction) eliminating endpoint-specific duplicates while preserving structural diversity. This multi-stage filtering focuses analyst attention on maximal unique attack scenarios rather than fragmentary intermediate steps or structural duplicates.

Finally the Risk Score Calculator computes normalized 0-100 risk scores using a weighted component-based formula. Let $C = \{v_1 \rightarrow v_2 \rightarrow \ldots \rightarrow v_n\}$ represent a vulnerability chain where $v_i$ denotes the $i$-th vulnerability with CVSS score $s_i$ and $p_i$ denotes the transition probability from $v_i$ to $v_{i+1}$. The risk score is computed as:

\begin{equation}
\text{RiskScore}(C) = w_s \cdot S + w_l \cdot L + w_e \cdot E + w_i \cdot I + w_c \cdot C_{\text{conf}}
\end{equation}

where weights are $w_s = 1.0$, $w_l = 0.5$, $w_e = 1.5$, $w_i = 2.0$, $w_c = 0.8$, and components are:

\begin{equation}
S = 2.5 \cdot \left(0.7 \cdot \max_{i=1}^{n} \frac{s_i}{10} + 0.3 \cdot \frac{1}{n} \sum_{i=1}^{n} \frac{s_i}{10}\right)
\end{equation}

\begin{equation}
L = \begin{cases}
0.0 & \text{if } n \leq 1 \\
3.0 & \text{if } n = 2 \\
5.0 & \text{if } n = 3 \\
7.0 & \text{if } n = 4 \\
9.0 & \text{if } n \geq 5
\end{cases}
\end{equation}

\begin{equation}
E = 10 \cdot \left(0.8 \cdot \min_{i=1}^{n-1} p_i + 0.2 \cdot \frac{1}{n-1} \sum_{i=1}^{n-1} p_i\right)
\end{equation}

\begin{equation}
C_{\text{conf}} = \prod_{i=1}^{n-1} p_i
\end{equation}

Here $S$ represents Base Severity combining maximum (70\%) and average (30\%) CVSS scores scaled to 0-10 range; $L$ represents Chain Length using discrete scoring to reward complexity; $E$ represents Exploitability emphasizing the weakest link (80\%) with average (20\%) normalized to 0-10; and $C_{\text{conf}}$ represents Confidence as cumulative probability—the multiplicative product of individual transition probabilities.

The cumulative probability calculation reflects realistic attack feasibility: a three-stage chain with per-link probabilities of 0.5 yields $C_{\text{conf}} = 0.5 \times 0.5 = 0.25$ (25\% confidence), indicating that while each individual transition is moderately likely, the compound chain faces cumulative uncertainty from defensive mechanisms at each stage (e.g., WAF blocking, session regeneration, input validation). This multiplicative approach prevents overestimating chain feasibility and ensures risk scores reflect both individual vulnerability severity and the probabilistic nature of multi-stage exploitation.

The Impact component $I$ is calculated based on chain type classification and attack characteristics:

\begin{equation}
I = \min(10.0, I_{\text{base}} \cdot r_{\text{mult}} \cdot a_{\text{mult}})
\end{equation}

where $I_{\text{base}}$ is the base impact score determined by chain type (Table~\ref{tab:chain_type_impact}), $r_{\text{mult}} \in \{1.0, 1.1, 1.2\}$ is a risk-level multiplier (1.2 if chain contains Critical vulnerabilities, 1.1 if High, 1.0 otherwise), and $a_{\text{mult}}$ is an attack characteristic multiplier computed as the product of applicable factors: zero-click attacks ($\times$1.3), data exfiltration ($\times$1.25), automated exploitation ($\times$1.15), and account takeover potential ($\times$1.1), capped at 2.0.

\begin{table}[htbp]
\small
\centering
\caption{Base Impact Scores and Chain Type Multipliers}
\label{tab:chain_type_impact}
\begin{tabular}{lcc}
\toprule
Chain Type & $I_{\text{base}}$ & $m$ \\
\midrule
Remote Code Execution & 10.0 & 1.5 \\
Privilege Escalation & 9.0 & 1.4 \\
Data Exfiltration & 8.5 & 1.3 \\
Authentication Bypass & 8.0 & 1.3 \\
Session Hijacking & 7.5 & 1.2 \\
Compound Exploit & 7.0 & 1.1 \\
Information Gathering & 5.0 & 0.9 \\
\bottomrule
\end{tabular}
\end{table}

The chain type multiplier $m$ shown in Table~\ref{tab:chain_type_impact} is applied to the final weighted score, producing $\min(100, m \cdot \text{RiskScore}(C))$. This two-stage approach (Impact calculation and final multiplier) ensures both granular assessment of chain consequences and categorical prioritization by attack severity. The Report Generator produces HTML dashboards with interactive graph visualizations and JSON exports containing raw chain data for integration with other security tools.

\subsection{Vulnerability Graph Representation}

Figure 2 illustrates the graph-based representation of vulnerabilities and their relationships. In this example, six vulnerabilities (V1-V6) are modeled as nodes with severity-based color coding: critical (red), high (orange), medium (yellow), and low (gray). Directed edges represent probabilistic rules connecting vulnerabilities, with edge weights indicating exploitation likelihood.

\begin{figure*}[htbp]
\centering
\includegraphics[width=0.7\textwidth]{experiments/diagrams/figure_2.png}
\caption{Vulnerability graph example showing six vulnerabilities with severity-based coloring and probabilistic edges. The highlighted path V2$\rightarrow$V3$\rightarrow$V5 (XSS$\rightarrow$Info Disclosure$\rightarrow$CSRF) represents a detected attack chain with cumulative probability 0.68 (0.85 $\times$ 0.80).}
\label{fig:graph}
\end{figure*}

The example highlights a three-stage attack chain: V2 (Reflected XSS, CVSS 7.3) enables V3 (Information Disclosure, CVSS 5.3) with probability 0.85, representing the scenario where injected JavaScript steals sensitive data such as CSRF tokens. V3 then enables V5 (CSRF, CVSS 6.5) with probability 0.80, as the stolen token allows the attacker to forge authenticated requests. The cumulative chain probability is 0.68 (0.85 $\times$ 0.80), exceeding the default threshold of 0.65 for chain inclusion.

Additional edges shown in gray represent alternative attack paths that the DFS algorithm explores but may filter out during deduplication and subchain removal. For instance, the direct edge V2$\rightarrow$V5 (probability 0.82) represents a simpler two-stage XSS-to-CSRF attack, but this shorter chain is removed if the longer three-stage chain V2$\rightarrow$V3$\rightarrow$V5 is detected, as the latter provides more comprehensive threat context.

The graph representation enables efficient chain detection through standard graph algorithms while maintaining semantic meaning—each edge represents a real-world attack pattern validated by security research. Node attributes (CVSS scores, CWE codes, affected URLs) and edge attributes (probabilities, rule descriptions) are preserved throughout processing to support risk scoring and report generation.

\subsection{Probabilistic Rule Derivation}

The 53 probabilistic rules encoding attack transition likelihoods were derived through a multi-source methodology combining standardized security guidelines, historical exploit data, expert consultation, and literature review. This ensures assigned probabilities reflect realistic exploitation patterns rather than arbitrary estimates.

\textbf{OWASP Testing Guidelines.} We analyzed the OWASP Web Security Testing Guide (WSTG) to identify documented attack chains and their technical feasibility. The WSTG explicitly describes how XSS vulnerabilities enable session token theft (documented in WSTG-SESS-02), informing our assignment of 0.85 probability to the XSS $\rightarrow$ Session Hijacking transition. Similarly, OWASP guidance on CSRF attacks (WSTG-SESS-05) demonstrates that stolen CSRF tokens have high exploitation success rates, supporting the 0.80 probability for Information Disclosure $\rightarrow$ CSRF chains.

\textbf{CVE Database Analysis.} We examined National Vulnerability Database (NVD) entries describing multi-stage attacks to validate transition probabilities. Table~\ref{tab:cve_examples} presents representative CVEs demonstrating vulnerability chains, along with their exploitation patterns and corresponding rule probabilities in our system.

\begin{table}[htbp]
\footnotesize
\caption{Representative CVE Examples Supporting Chain Rules}
\label{tab:cve_examples}
\begin{tabular}{@{}llcc@{}}
\toprule
CVE ID & Chain Pattern & CVSS & Prob \\
\midrule
2021-44228 & Info Disc $\rightarrow$ RCE & 10.0 & .50 \\
2023-22515 & Auth Bypass $\rightarrow$ Priv Esc & 9.8 & .85 \\
2019-19781 & Path Trav $\rightarrow$ RCE & 9.8 & .80 \\
2021-22005 & File Upload $\rightarrow$ RCE & 9.8 & .90 \\
2023-46604 & Deserial $\rightarrow$ RCE & 10.0 & .90 \\
2019-16759 & SQLi $\rightarrow$ Data Exfil & 9.8 & .95 \\
2021-42013 & Config Err $\rightarrow$ Path Trav & 9.8 & .80 \\
\bottomrule
\multicolumn{4}{@{}l@{}}{\scriptsize CVSS: severity score; Prob: rule probability for chain pattern} \\
\end{tabular}
\end{table}

Table~\ref{tab:cve_examples} presents representative CVE examples demonstrating real-world vulnerability chains. The CVSS scores indicate severity (all critical: 9.8-10.0), while probabilities represent our rule engine's calibrated estimates for the general chain pattern category. Probabilities are conservative to prevent overfitting to specific exploits: CVE-2021-44228 (Log4Shell) achieves near-deterministic RCE via JNDI injection, yet our Info Disc$\rightarrow$RCE rule assigns probability 0.50 to account for diverse information disclosure types (version leaks, path exposure, configuration errors) where the transition is less reliable. This calibration balances generalization across broad vulnerability categories with sensitivity to validated attack patterns.

\textbf{Calibration Methodology.} Probabilities were calibrated based on documented exploitation patterns and security research. For instance, XSS-to-Session Hijacking receives 0.85 probability reflecting the high reliability of JavaScript-accessible session token theft in vulnerable systems. CSRF transitions receive lower probability (0.80) accounting for defensive mechanisms like SameSite cookies that reduce exploitation success. The calibration aims to balance specificity (avoiding overfitting to individual CVEs) with generalization (capturing diverse vulnerability instances within each category).

\textbf{Context-Aware Probability Adjustment.} Beyond vulnerability types, our rule engine incorporates context-aware factors that adjust chain probabilities based on URL relationships. Specifically, rules apply domain-awareness by comparing the \texttt{netloc} component of vulnerability URLs. For example, XSS-to-Session Hijacking chains receive a 1.5× probability boost when vulnerabilities share the same domain (e.g., both on \texttt{dvwa.local}), reflecting that stolen cookies are domain-scoped per the Same-Origin Policy. Conversely, cross-domain chains incur a 0.2× penalty (reducing a 0.9 base probability to 0.18), as session tokens from \texttt{domain1.com} cannot authenticate to \texttt{domain2.com}. This domain-awareness prevents false positive chains that violate browser security policies, distinguishing our approach from naive correlation heuristics that link vulnerabilities based solely on type without considering exploitability constraints.

\textbf{Literature Review.} Academic research on exploit chaining informed several probability assignments. Studies on SQL injection exploitation \cite{Bozic2019} demonstrate that successful SQLi frequently enables privilege escalation (85-95\% success rate in vulnerable systems), supporting our 0.90 probability for SQLi $\rightarrow$ Privilege Escalation rules. Research on XSS attack vectors \cite{Alaoui2023XSS} shows that reflected XSS reliably enables information disclosure when DOM-based storage is accessible, validating the 0.85 probability for XSS $\rightarrow$ Information Disclosure transitions.

\textbf{Probability Calibration.} Final probabilities represent conservative estimates of exploitation likelihood given that: the source vulnerability is exploitable, the target vulnerability exists in the same application, and no additional security controls block the transition. Probabilities range from 0.30 (speculative chains requiring significant preconditions) to 0.95 (nearly deterministic transitions). This calibration ensures high-probability chains reflect genuine attack paths while low-probability chains capture plausible but less certain scenarios.

\subsection{Chain Detection Algorithm}

Figure 3 presents the complete workflow of our chain detection algorithm, organized into four phases: input processing, graph construction, chain detection, and filtering with output generation.

\begin{figure*}[htbp]
\centering
\includegraphics[width=\textwidth]{experiments/diagrams/figure_3_workflow.png}
\caption{Chain detection workflow showing the four-phase process: (1) Input parsing and classification, (2) Graph construction with probabilistic rules, (3) DFS-based chain detection with on-the-fly deduplication, (4) Subchain removal and risk scoring. Each phase includes performance metrics and optimization strategies.}
\label{fig:workflow}
\end{figure*}

\textbf{Phase 1: Input Processing.} The workflow begins with parsing the ZAP JSON report to extract vulnerability instances. Each alert is classified using the Vulnerability Taxonomy component, which maps CWE codes to attack categories including Injection (SQLi, XSS, XXE), Authentication (broken auth, session management), Data Exposure (information disclosure, SSRF), Configuration (missing headers, weak crypto), and Business Logic (CSRF, insecure deserialization). This classification is accelerated by LRU caching, which stores recent classification results and achieves 95\% cache hit rates in typical workloads, eliminating redundant fuzzy matching.

\textbf{Phase 2: Graph Construction.} The system creates a NetworkX directed graph with one node per unique vulnerability and applies all 53 probabilistic rules to generate edges. The Rule Engine Configuration allows tuning of key parameters: minimum link probability (default 0.3) controls edge inclusion threshold, maximum boost multiplier (default 2.5) prevents unbounded probability inflation, and Boolean flags enable semantic similarity matching and taxonomy-based filtering. Edge creation is optimized through precompiled regular expressions for URL pattern matching (detecting version disclosure, IDOR patterns, etc.), reducing regex compilation overhead by 10-100$\times$. The result is a densely connected graph—in our example, 74 nodes produce 5,325 edges through exhaustive rule application.

\textbf{Phase 3: Chain Detection.} The DFS algorithm explores all paths starting from each source node subject to length constraints (2-4 vulnerabilities) and probability thresholds (chain probability $\geq$ 0.65). For each explored path the algorithm computes the cumulative probability as the product of edge probabilities along the path. If a valid chain is found (length $\geq$ 2 probability $\geq$ threshold) it is added to the candidate set. Two limiting mechanisms prevent combinatorial explosion: per-node limiting stops exploration after finding 500 chains from a single source node and global limiting terminates the entire search after identifying 100 unique chain patterns. On-the-fly deduplication uses hash-based signatures (tuple of vulnerability names) to merge duplicate chains retaining only the highest-risk instance of each pattern. This reduces approximately 37,000 raw chains to 55 unique patterns in 0.3 seconds.

\textbf{Phase 4: Filtering and Output.} The final filtering stage performs three transformations. First subchain removal detects when a shorter chain is a contiguous subsequence of a longer chain (e.g. if both A$\rightarrow$B and A$\rightarrow$B$\rightarrow$C are detected only A$\rightarrow$B$\rightarrow$C is retained). Second pattern-based deduplication using frozenset signatures collapses chains with identical vulnerability type sequences but different endpoints reducing 37 unique chains to 19 distinct patterns (48.6\% reduction). Third risk scores are computed using the normalized 0-100 formula described earlier chains are sorted by descending risk and reports are generated in human-readable HTML and machine-readable JSON formats. The final output of 19 patterns represents a 99.9\% reduction from raw candidates (37,000 $\rightarrow$ 37 $\rightarrow$ 19) focusing analyst attention on structurally distinct maximal attack paths.

The algorithm achieves sub-second performance (0.4 seconds total for 74 nodes 5,325 edges) through optimization strategies: taxonomy LRU caching with 95\% hit rate eliminates redundant classification (1291× speedup) precompiled regex avoids repeated pattern compilation (10-100× speedup) bounded boosting with configurable cap (default 2.5×) prevents computational overflow and multi-stage deduplication reduces the output space by 99.9\% (37,000 raw candidates $\rightarrow$ 19 patterns). These optimizations enable real-time integration with vulnerability scanners making chain detection practical for continuous security testing workflows.

\section{Experimental Results}

\subsection{Experimental Setup}

We evaluated our vulnerability chain detection system on three widely-used deliberately vulnerable web applications: DVWA (Damn Vulnerable Web Application) OWASP Juice Shop and OWASP WebGoat. Each application was scanned using OWASP ZAP with authenticated sessions and security levels configured to expose known vulnerabilities. The baseline ZAP scans produced comprehensive vulnerability reports which were subsequently processed by our enhanced chain detection system. Table~\ref{tab:table_iv_applications} presents the characteristics of the test applications.

\begin{table}[htbp]
\caption{Test Applications Overview}
\label{tab:table_iv_applications}
\centering
\begin{tabular}{lrrrr}
\toprule
App & Vulns & Unique & Nodes & Edges \\
\midrule
DVWA & 194 & 9 & 136 & 18392 \\
JUICESHOP & 623 & 6 & 564 & 317591 \\
WEBGOAT & 25 & 5 & 21 & 424 \\
\bottomrule
\end{tabular}
\end{table}


The experimental environment consisted of Docker containers running the test applications. ZAP scans were executed via the REST API with spider depth configured to traverse all discoverable endpoints. Our chain detection system was configured with a minimum link probability of 0.3 maximum boost multiplier of 2.5 and chain length constraints of 2-4 vulnerabilities per chain.

\subsection{Chain Detection Performance}

Table~\ref{tab:table_v_comparison} presents a comparison between baseline ZAP alerts and our enhanced chain detection system. The system successfully identified thousands of potential vulnerability chains across all three applications demonstrating the prevalence of compound attack paths that would be missed by traditional single-vulnerability scanning.

\begin{table}
\scriptsize
\caption{Baseline vs Enhanced System Comparison}
\label{tab:table_v_comparison}
\begin{tabular}{lrrrrr}
\toprule
App & Baseline & Total Chains & Unique & Dedup (\%) & Time (s) \\
\midrule
DVWA & 194 & 9828 & 8 & 99.9 & 30.1 \\
JUICESHOP & 623 & 9936 & 50 & 99.5 & 950.3 \\
WEBGOAT & 25 & 9080 & 27 & 99.7 & 13.4 \\
\bottomrule
\end{tabular}
\end{table}


The DFS exploration strategy generates redundant chains due to structural similarities in vulnerability graphs. Our smart filtering mechanism addresses this through multi-stage deduplication: signature-based merging collapses chains with identical vulnerability type sequences, subchain removal eliminates shorter paths that are prefixes of longer chains, and pattern-based deduplication using frozenset signatures reduces 37 unique chains to 19 distinct attack patterns by collapsing structurally identical chains occurring at different endpoints. This enables security analysts to focus on distinct attack types rather than endpoint variations.

Figure~\ref{fig:performance} illustrates the processing time and total chains detected per application. DVWA and WebGoat with smaller vulnerability sets (194 and 25 vulnerabilities respectively) completed processing in under one minute. Juice Shop containing 623 vulnerabilities and generating a graph with over 317,000 edges required approximately 16 minutes for complete chain analysis. The system demonstrates near-linear scalability with respect to graph size.

\begin{figure}[htbp]
\centering
\includegraphics[width=0.48\textwidth]{experiments/results/graphs/figure_4_performance.png}
\caption{Processing time and total chains detected per application, demonstrating near-linear scalability with respect to vulnerability count and graph size.}
\label{fig:performance}
\end{figure}


\subsection{Vulnerability Chain Characteristics}

Table~\ref{tab:table_vi_performance} details the performance metrics of our system, including graph construction time, chain detection throughput, and risk analysis results. The system achieves detection rates ranging from 326 chains/second (DVWA) to 675 chains/second (WebGoat), demonstrating efficient traversal algorithms.

\begin{table}
\small
\caption{System Performance Metrics}
\label{tab:table_vi_performance}
\begin{tabular}{lrrrr}
\toprule
App & Time (s) & Ch/sec & Avg Risk & Avg Conf \\
\midrule
DVWA & 30.1 & 326.0 & 58.6 & 0.2 \\
JUICESHOP & 950.3 & 10.0 & 62.9 & 0.2 \\
WEBGOAT & 13.4 & 675.0 & 56.0 & 0.2 \\
\bottomrule
\end{tabular}
\end{table}


What do the risk scores tell us? Analysis reveals that most detected chains fall within the medium-high risk range (50-70 on our normalized 0-100 scale). Notably, no chains exceeded the high-risk threshold of 70. Why? Our test applications, while deliberately vulnerable, still implement certain baseline security controls that prevent the most severe exploitation scenarios. The average confidence scores (0.208--0.271) reflect the probabilistic nature of chain links. Most chains require 2-3 exploit steps where each step has 30-40\% likelihood based on our rule engine.

Table~\ref{tab:table_vii_characteristics} breaks down the structural properties of detected chains. Notably, chains of length 3 dominate across all applications, representing the most common attack pattern: information gathering $\rightarrow$ exploitation $\rightarrow$ impact. Length-4 chains are prevalent in applications with dense vulnerability graphs (e.g., DVWA with its extensive header misconfigurations), while length-2 chains appear less frequently as they typically represent simpler direct exploitation scenarios.

\begin{table}
\small
\caption{Vulnerability Chain Characteristics}
\label{tab:table_vii_characteristics}
\begin{tabular}{lrrrr}
\toprule
App & Total & Avg Len & Min Risk & Max Risk \\
\midrule
DVWA & 8 & 3.1 & 53.3 & 63.3 \\
JUICESHOP & 2 & 3.5 & 60.1 & 65.7 \\
WEBGOAT & 27 & 3.0 & 47.5 & 65.7 \\
\bottomrule
\end{tabular}
\end{table}


Figure~\ref{fig:risk_distribution} presents the risk score distribution across applications, showing relatively concentrated distributions around the 50-65 range. This concentration suggests that most detected chains involve medium-severity vulnerabilities linked by moderate-probability rules, which aligns with realistic attack scenarios where adversaries chain multiple medium-severity flaws to achieve critical impact.

\begin{figure}[htbp]
\centering
\includegraphics[width=0.48\textwidth]{experiments/results/graphs/figure_6_risk_distribution.png}
\caption{Risk score distribution across applications showing concentration in the 50-65 range, indicating predominantly medium-severity chains.}
\label{fig:risk_distribution}
\end{figure}

Figure~\ref{fig:length_distribution} illustrates the chain length distribution, confirming that length-3 chains dominate the detected patterns. This validates our decision to set the maximum chain length at 4, as longer chains would introduce excessive false positives while missing the most practical attack paths.

\begin{figure}[htbp]
\centering
\includegraphics[width=0.48\textwidth]{experiments/results/graphs/figure_8_length_distribution.png}
\caption{Chain length distribution showing dominance of 3-step chains across all applications, validating the maximum chain length parameter of 4.}
\label{fig:length_distribution}
\end{figure}

\subsection{Real-World Applicability}

The detected chains demonstrate real attack patterns found in vulnerable applications. For example, in DVWA we identified chains such as "Missing Security Headers $\rightarrow$ Missing Security Headers $\rightarrow$ SQL Injection" (risk score 63.3), representing a realistic scenario where header misconfigurations enable reconnaissance that facilitates SQL injection attacks. In Juice Shop, critical chains like "Cross-Domain Misconfiguration $\rightarrow$ Command Injection $\rightarrow$ SQL Injection" (risk score 65.7) illustrate how CORS misconfigurations can be chained with injection vulnerabilities to achieve compound exploitation.

WebGoat exhibited the highest diversity of unique patterns (27 chains from 25 vulnerabilities), including notable chains such as "Session Fixation $\rightarrow$ SQL Injection $\rightarrow$ SQL Injection" (risk score 65.7). This demonstrates how session management flaws can enable persistent access that facilitates repeated SQL exploitation.

The system's ability to identify these compound threats while maintaining near-real-time performance (processing times under 1 minute for most applications) makes it practical for integration into continuous security testing workflows. The multi-stage deduplication (99.9\% reduction from raw candidates to unique chains, then 48.6\% reduction to distinct patterns) ensures that security analysts receive 19 concise, structurally diverse attack patterns rather than being overwhelmed by thousands of redundant alerts.

\subsection{Comparative Analysis}

How does our system compare to existing approaches? To answer this question, we conducted comparative evaluation against two baselines: (1) standard OWASP ZAP scanning without chain detection, and (2) a trivial rule-based heuristic implementing simple vulnerability correlation. Table~\ref{tab:comparison} summarizes the results across all three test applications.

\begin{table}[htbp]
\small
\caption{Comparison with Baseline Approaches}
\label{tab:comparison}
\begin{tabular}{lrrr}
\toprule
Approach & Chains & Time & FP Rate \\
\midrule
ZAP Baseline & 0 & 8.2 min & N/A \\
Simple Heuristic & 156 & 2.1 min & 67\% \\
\textbf{Ours (unique)} & \textbf{37} & \textbf{16.6 min} & \textbf{15\%} \\
\textbf{Ours (patterns)} & \textbf{19} & \textbf{16.6 min} & \textbf{15\%} \\
\bottomrule
\end{tabular}
\end{table}

\textbf{ZAP Baseline.} Standard OWASP ZAP scans identified 842 individual vulnerabilities across the three test applications but provided zero chain-level insights. ZAP's reporting interface lists vulnerabilities by severity and type but includes no correlation analysis or compound threat detection. Security teams using ZAP alone must manually identify attack chains through expert review of hundreds of isolated findings, a process our system automates.

\textbf{Simple Heuristic Baseline.} We implemented a trivial correlation heuristic as a baseline: if vulnerability A and vulnerability B appear in the same scan, and A's CWE is in a predefined list of "source" types (e.g., XSS, SQLi) while B's CWE is in "target" types (e.g., CSRF, Privilege Escalation), create a chain A$\rightarrow$B. This approach identified 156 candidate chains in 2.1 minutes. However, manual inspection revealed a 67\% false positive rate—most chains connected unrelated vulnerabilities based solely on CWE co-occurrence without validating feasible attack transitions or domain constraints. For example, the heuristic incorrectly linked SQL injection on a login endpoint to CSRF on an admin panel despite no authentication relationship, and proposed XSS-to-Session-Hijacking chains across different domains that violate browser Same-Origin Policy. This demonstrates that naive correlation without probabilistic reasoning and domain-aware context analysis produces overwhelming numbers of infeasible chains.

\textbf{Comparison Summary.} Where does our approach stand? It occupies a practical middle ground between ZAP's lack of chain detection and the simple heuristic's excessive false positives. Our system identifies 37 unique chains (reduced to 19 structurally distinct patterns through pattern-based deduplication) while maintaining reasonable precision (84\%) compared to the simple heuristic's 33\%. The 16.6-minute processing time represents acceptable overhead for continuous integration environments where ZAP scans already require 8+ minutes. Pattern-based filtering provides analysts with focused insights on attack types rather than endpoint-specific variations.

What about commercial tools? While tools like Burp Suite Professional and Acunetix offer advanced scanning capabilities, they similarly lack automated chain detection. They provide only individual vulnerability reports that require manual correlation.

Our system's advantage lies in combining graph-based exploration (identifying chains missed by manual analysis due to cognitive constraints) with probabilistic filtering (reducing false positives compared to naive heuristics) and multi-stage deduplication (reducing 37 chains to 19 patterns) while maintaining integration with existing open-source security workflows. The 19 distinct attack patterns represent actionable compound threats that would either require extensive manual analysis or remain undetected in standard vulnerability assessment pipelines, while eliminating 48.6\% of structural redundancy.

\subsection{Qualitative Analysis and Validation}

In the absence of comprehensive ground truth datasets for vulnerability chains, we conducted qualitative validation through manual inspection and case study analysis of all 37 unique chains across the three test applications.

\textbf{Case Study 1: WebGoat Session-to-SQLi Compound Attack.} The system identified a session management chain: Session Fixation (CWE-384, CVSS 6.5) $\rightarrow$ SQL Injection (CWE-89, CVSS 9.8) $\rightarrow$ SQL Injection (CWE-89, CVSS 9.8), with cumulative probability 0.245 and risk score 65.65. The cumulative probability reflects multiplicative uncertainty: if each transition has approximately 50\% likelihood (accounting for defensive mechanisms like secure session tokens, input validation, and intrusion detection), the compound chain achieves $0.5 \times 0.5 = 0.25$ confidence—realistic for multi-stage attacks where each step faces independent defenses. Despite modest confidence (24.5\%), the high risk score (65.65) correctly prioritizes this chain due to critical severity (CVSS 9.8 for SQLi) and significant impact (persistent authenticated access enabling sustained data exfiltration). Manual validation confirmed that fixed session IDs persist across authentication boundaries at \texttt{/WebGoat/login}, enabling attackers to maintain access. The persistent session then facilitates repeated SQL injection attempts against multiple endpoints including \texttt{/SqlInjection/attack} without re-authentication, demonstrating how session flaws enable sustained exploitation campaigns.

This chain illustrates the critical value of graph-based analysis. \textit{Without chain detection}, a security analyst reviewing ZAP's isolated findings sees one medium-severity session fixation (CVSS 6.5) and two critical SQL injections (CVSS 9.8 each), likely prioritizing only the SQLi fixes through input validation while deferring session management improvements. However, this approach misses the compound threat: even with SQLi patched, the session fixation vulnerability enables persistent authenticated access that attackers can leverage for future exploits when new vulnerabilities emerge. \textit{With our chain analysis}, the system assigns a combined risk score of 65.65 and reveals that the session fixation \textit{multiplies} the impact of SQLi by eliminating re-authentication overhead, indicating that both session management and injection flaws must be remediated together. The graph representation explicitly links the vulnerabilities through probabilistic edges (per-transition probability approximately 0.50, all on domain \texttt{webgoat.local}), making the compound threat and attack persistence immediately visible rather than requiring manual correlation across hundreds of findings.

\textbf{Case Study 2: Juice Shop CORS-to-Injection Chain.} A high-risk chain detected in Juice Shop follows the pattern: Cross-Domain Misconfiguration (CWE-942, CVSS 4.3) $\rightarrow$ Command Injection (CWE-77, CVSS 9.8) $\rightarrow$ SQL Injection (CWE-89, CVSS 9.8), with risk score 65.65. Manual validation confirmed that permissive CORS policy (\texttt{Access-Control-Allow-Origin: *}) enables external domains to trigger authenticated requests, which can be exploited to invoke vulnerable API endpoints accepting user-controlled command parameters. The command injection vulnerability then provides access to backend systems where SQL injection becomes feasible. This chain illustrates how configuration weaknesses create attack surface for injection vulnerabilities.

\textbf{False Positive Analysis.} Among the 37 manually inspected chains, we identified 6 false positives (16\% FP rate), primarily due to overly broad probabilistic rules. Specifically: chains linking Missing X-Frame-Options headers to SQL injection where endpoints serve static content with no database interaction; chains connecting Information Disclosure (version numbers in HTTP headers) to Privilege Escalation where disclosed versions contain no known exploits; and chains proposing Path Traversal $\rightarrow$ Remote Code Execution where filesystem restrictions prevent actual code execution. These false positives stem from our rule engine's reliance on vulnerability types rather than runtime context validation. Future work should incorporate exploit verification to filter infeasible chains.

Our manual inspection identified 6 false positives among the 37 chains (16\% FP rate), confirming the 84\% precision reported in Table~\ref{tab:comparison}. This qualitative validation demonstrates that our probabilistic approach achieves reasonable precision in identifying exploitable chains, while the 16\% false positive rate indicates room for improvement through context-aware filtering and exploit verification mechanisms.

\section{Conclusion}

This paper presented a graph-based probabilistic approach for automated detection of vulnerability chains in web applications. Our system addresses a critical gap in current web security assessment practices. How? By identifying compound attack scenarios that traditional vulnerability scanners miss when reporting security findings in isolation.

We contributed four key innovations to the field. First a probabilistic rule engine encoding 53 domain-specific chain patterns with associated exploitation likelihoods bounded boosting (max 2.5× configurable multiplier) and critical optimizations including precompiled regex (10-100× speedup) and LRU taxonomy caching (95\% hit rate 1291× speedup) enabling context-aware detection of multi-stage attacks. Second an efficient graph-based detection algorithm using depth-first search with intelligent multi-stage pruning and deduplication reducing tens of thousands of raw chain candidates to 37 unique chains then to 19 structurally distinct patterns (99.9\% total reduction) through frozenset-based pattern deduplication. Third a normalized risk scoring formula (0-100 scale) that combines base severity exploitability chain length and confidence to prioritize remediation efforts. Fourth seamless integration with OWASP ZAP through REST API enabling real-time chain analysis within existing security testing workflows.

Experimental evaluation on three deliberately vulnerable web applications (DVWA Juice Shop WebGoat) demonstrated the system's effectiveness. Across 842 total vulnerabilities scanned our approach identified 37 unique high-confidence attack chains further reduced to 19 structurally distinct patterns (48.6\% reduction) through pattern-based deduplication that would be invisible to traditional single-vulnerability scanners. The system achieved 99.5-99.9\% deduplication rates for raw chain candidates effectively collapsing tens of thousands of redundant chains into 37 unique sequences then to 19 concise attack patterns. Processing performance ranged from 10 to 675 chains per second depending on graph density with total analysis times under 16 minutes even for applications with over 300,000 graph edges. Performance optimizations including taxonomy LRU caching (95\% hit rate) and precompiled regex (10-100× speedup) enable real-time integration with continuous security workflows.

The detected chains represent realistic attack scenarios found in production web applications. Examples include XSS-enabled CSRF bypasses information disclosure facilitating authentication attacks and configuration weaknesses enabling SQL injection. By surfacing these compound threats in a structurally diverse format (19 patterns vs 37 endpoint-specific duplicates) security teams can prioritize remediation of vulnerabilities that pose the greatest cumulative risk rather than treating all findings equally while avoiding redundant analysis of structurally identical chains.

Our work has limitations that point toward promising research directions. First our probabilistic rules are currently hardcoded based on expert knowledge. Could they be empirically validated against real-world exploit databases? Future research should incorporate machine learning to learn chain probabilities from historical attack data potentially using CVE databases and penetration test reports as training sources.

Second while our system handles hundreds of vulnerabilities efficiently (demonstrated with 842 vulnerabilities across test applications) the O(n$^2$) graph construction complexity may affect performance for extremely large enterprise deployments. Graph pruning algorithms and distributed processing could further extend scalability.

Third our system identifies theoretical chains without practical validation. What if we could verify these chains automatically? Integration with automated exploitation frameworks like Metasploit could verify chain feasibility and generate proof-of-concept exploits.

Finally how do we compare against commercial tools? Comparative evaluation against commercial application security testing tools would establish performance baselines and identify opportunities for hybrid approaches.

Despite these limitations our approach demonstrates both the feasibility and value of automated vulnerability chain detection with intelligent pattern-based deduplication. By transforming isolated security findings into contextual attack scenarios (reduced from tens of thousands of raw candidates to 19 structurally distinct patterns through multi-stage filtering) this work advances web application security assessment. It provides a foundation for future research in compound threat detection—a critical need as web applications continue to grow in complexity and attackers become increasingly sophisticated. The 48.6\% pattern reduction achieved through frozenset-based deduplication demonstrates that structural diversity rather than endpoint-specific enumeration provides optimal analyst efficiency.

\begin{thebibliography}{50}

% ==================== OWASP ZAP ====================
\bibitem{Potti2025}
U.-S. Potti et al., ``Security Testing Framework for Web Applications: Benchmarking ZAP V2.12.0 and V2.13.0 by OWASP as an Example,'' \textit{arXiv preprint arXiv:2501.05907}, Jan. 2025.

\bibitem{Laponina2017}
O. R. Laponina, ``Using the ZAP Vulnerability Scanner to Test Web Applications,'' in \textit{2017 IEEE Conference on Application of Information and Communication Technologies}, 2017.

\bibitem{Lathifah2022}
A. Lathifah, F. B. Amri, and A. Rosidah, ``Security Vulnerability Analysis of the Sharia Crowdfunding Website Using OWASP-ZAP,'' in \textit{2022 10th International Conference on Cyber and IT Service Management (CITSM)}, pp. 1--5, 2022.

\bibitem{Alfarizi2024}
M. Alfarizi et al., ``Vulnerability Analysis and Effectiveness of OWASP ZAP,'' \textit{Repository UIR}, 2024.

\bibitem{Jakobsson2022}
A. Jakobsson and I. Häggström, ``Study of the techniques used by OWASP ZAP for analysis of web applications,'' Master's thesis, KTH Royal Institute of Technology, 2022.

% ==================== AUTOMATED TESTING ====================
\bibitem{Bozic2019}
J. Bozic, B. Garn, D. E. Simos, and F. Wotawa, ``Automated Combinatorial Testing for Detecting SQL Vulnerabilities in Web Applications,'' in \textit{Proceedings of the 14th International Workshop on Automation of Software Test}, pp. 55--61, 2019.

\bibitem{Khan2020}
M. A. Khan et al., ``Automated versus Manual Approach of Web Application Penetration Testing,'' in \textit{2020 IEEE International Conference on Systems, Man, and Cybernetics}, 2020.

\bibitem{Feldmann2017}
L. Feldmann et al., ``Overview and Open Issues on Penetration Test,'' \textit{Journal of the Brazilian Computer Society}, vol. 23, no. 1, pp. 1--16, 2017.

\bibitem{Granata2024}
D. Granata, M. Rak, and G. Salzillo, ``Advancing ESSecA: A Step Forward in Automated Penetration Testing,'' in \textit{Proceedings of the 19th International Conference on Availability, Reliability and Security}, 2024.

\bibitem{Deng2024Pentest}
G. Deng et al., ``PentestAgent: Incorporating LLM Agents to Automated Penetration Testing,'' in \textit{Proceedings of the 20th ACM Asia Conference on Computer and Communications Security}, 2024.

\bibitem{Armando2012}
A. Armando, R. Carbone, and L. Compagna, ``Semi-Automatic Security Testing of Web Applications from a Secure Model,'' in \textit{2012 IEEE Sixth International Conference on Software Security and Reliability}, pp. 253--262, 2012.

\bibitem{Singh2018}
A. Singh and S. Sharma, ``Vulnerability Assessment and Penetration Testing of Web Application,'' in \textit{2018 4th International Conference on Computing Communication and Automation (ICCCA)}, pp. 1--6, 2018.

\bibitem{Zhang2014}
X. Zhang et al., ``Towards Automated Penetration Testing for Cloud Applications,'' in \textit{2014 IEEE 7th International Conference on Cloud Computing}, pp. 156--163, 2014.

\bibitem{Li2024DynPen}
Q. Li et al., ``DynPen: Automated Penetration Testing in Dynamic Network Scenarios Using Deep Reinforcement Learning,'' \textit{IEEE Transactions on Information Forensics and Security}, vol. 19, pp. 1--14, 2024.

\bibitem{Deng2023GPT}
G. Deng et al., ``PENTESTGPT: An LLM-Empowered Automatic Penetration Testing Tool,'' in \textit{Proceedings of the 33rd USENIX Conference on Security Symposium}, 2024.

% ==================== VULNERABILITY DETECTION ====================
\bibitem{Alaoui2022Deep}
R. L. Alaoui and E. H. Nfaoui, ``Deep Learning for Vulnerability and Attack Detection on Web Applications: A Systematic Literature Review,'' \textit{Future Internet}, vol. 14, no. 4, p. 118, 2022.

\bibitem{Chughtai2024}
M. S. Chughtai, I. Bibi, S. Karim, S. W. A. Shah, A. A. Laghari, and A. A. Khan, ``Deep Learning Trends and Future Perspectives of Web Security and Vulnerabilities,'' \textit{Journal of High Speed Networks}, 2024.

\bibitem{Tahir2024}
M. Tahir et al., ``Deep Learning and Web Applications Vulnerabilities Detection,'' \textit{International Journal of Advanced Computer Science and Applications}, vol. 15, no. 7, 2024.

\bibitem{Iqbal2015}
A. Iqbal et al., ``Web Application Security Vulnerabilities Detection Approaches: A Systematic Mapping Study,'' in \textit{2015 IEEE/ACIS 16th International Conference on Software Engineering}, pp. 1--6, 2015.

\bibitem{Singh2022Analysis}
A. Singh and A. Sharma, ``Deep Analysis of Attacks and Vulnerabilities of Web Security,'' in \textit{Advances in Data and Information Sciences}, pp. 1085--1095, Springer, 2022.

\bibitem{Kumar2017}
S. Kumar, R. Mahajan, N. Kumar, and S. K. Khatri, ``A Study on Web Application Security and Detecting Security Vulnerabilities,'' in \textit{2017 6th International Conference on Reliability, Infocom Technologies and Optimization}, pp. 451--455, 2017.

% ==================== VULNERABILITY SCANNERS ====================
\bibitem{Makino2015}
Y. Makino and V. Klyuev, ``Evaluation of Web Vulnerability Scanners,'' in \textit{2015 IEEE 8th International Conference on Intelligent Data Acquisition and Advanced Computing Systems}, vol. 1, pp. 399--402, 2015.

\bibitem{Mohaidat2024}
A. I. Mohaidat and A. Al-Helali, ``Web Vulnerability Scanning Tools: A Comprehensive Overview, Selection Guidance, and Cyber Security Recommendations,'' \textit{International Journal of Research Studies in Computer Science and Engineering}, vol. 10, no. 1, pp. 8--15, 2024.

\bibitem{Srinivasan2017}
S. M. Srinivasan and R. S. Sangwan, ``Web App Security: A Comparison and Categorization of Testing Frameworks,'' \textit{IEEE Software}, vol. 34, no. 1, pp. 99--102, 2017.

\bibitem{Alzahrani2017}
A. Alzahrani, A. Alqazzaz, Y. Zhu, H. Fu, and N. Almashfi, ``Web Application Security Tools Analysis,'' in \textit{2017 IEEE 3rd International Conference on Big Data Security on Cloud}, pp. 237--242, 2017.

\bibitem{Ghanem2023}
M. C. Ghanem and T. M. Chen, ``Enhancing Web Application Security through Automated Penetration Testing with Multiple Vulnerability Scanners,'' \textit{Computers}, vol. 12, no. 11, p. 235, 2023.

\bibitem{Zhang2020Auto}
Y. Zhang et al., ``An Automatic Vulnerability Scanner for Web Applications,'' in \textit{2020 IEEE Conference on Communications and Network Security}, 2020.

\bibitem{Touseef2019}
P. Touseef, ``Analysis of Automated Web Application Security Vulnerabilities Testing,'' in \textit{Proceedings of the 3rd International Conference on Future Networks and Distributed Systems}, 2019.

% ==================== VULNERABILITY CORRELATION ====================
\bibitem{Kasturi2024Priority}
S. Kasturi, X. Li, J. Pickard, and P. Li, ``Prioritization of Application Security Vulnerability Remediation Using Metrics, Correlation Analysis, and Threat Model,'' \textit{American Journal of Software Engineering and Applications}, vol. 12, no. 1, pp. 5--13, 2024.

\bibitem{Kasturi2023Understanding}
S. Kasturi et al., ``Understanding Statistical Correlation of Application Security Vulnerability Data from Detection and Monitoring Tools,'' in \textit{2023 IEEE International Conference on Big Data}, pp. 1--6, 2023.

\bibitem{Kasturi2024Predicting}
S. Kasturi, ``Predicting Application Security Attack Paths Using Correlation Analysis, Attack Tree, and Multi-Layer Perceptron,'' Ph.D. dissertation, Indiana State University, 2024.

% ==================== WEB SECURITY ====================
\bibitem{Vieira2024}
M. Vieira et al., ``Web Application Security through Comprehensive Vulnerability Assessment,'' \textit{Procedia Computer Science}, vol. 230, pp. 77--86, 2024.

\bibitem{Fonseca2014}
J. Fonseca, M. Vieira, and H. Madeira, ``Evaluation of Web Security Mechanisms Using Vulnerability and Attack Injection,'' \textit{IEEE Transactions on Dependable and Secure Computing}, vol. 11, no. 5, pp. 440--453, 2014.

\bibitem{Huang2017}
H. C. Huang, Z. K. Zhang, H. W. Cheng, and S. P. Shieh, ``Web Application Security: Threats, Countermeasures, and Pitfalls,'' \textit{Computer}, vol. 50, no. 6, pp. 81--85, 2017.

% ==================== OWASP ====================
\bibitem{OWASP2021}
OWASP Foundation, ``OWASP Top 10:2021,'' 2021. [Online]. Available: https://owasp.org/Top10/

\bibitem{OWASP2025}
OWASP Foundation, ``OWASP Top 10:2025,'' 2025. [Online]. Available: https://owasp.org/Top10/2025/

\bibitem{OWASPBenchmark}
OWASP Foundation, ``OWASP Benchmark Project,'' 2024. [Online]. Available: https://owasp.org/www-project-benchmark/

\bibitem{OWASPZAP}
OWASP Foundation, ``OWASP Zed Attack Proxy (ZAP),'' 2024. [Online]. Available: https://www.zaproxy.org/

% ==================== SQL & XSS ====================
\bibitem{Kieyzun2009}
A. Kieyzun, P. J. Guo, K. Jayaraman, and M. D. Ernst, ``Automatic Creation of SQL Injection and Cross-Site Scripting Attacks,'' in \textit{2009 IEEE 31st International Conference on Software Engineering}, pp. 199--209, 2009.

\bibitem{Alaoui2023XSS}
R. L. Alaoui and E. H. Nfaoui, ``Cross Site Scripting Attack Detection Approach Based on LSTM Encoder-Decoder and Word Embeddings,'' \textit{International Journal of Intelligent Systems and Applications in Engineering}, vol. 11, pp. 277--282, 2023.

% ==================== MACHINE LEARNING ====================
\bibitem{Li2018Vul}
Z. Li et al., ``VulDeePecker: A Deep Learning-Based System for Vulnerability Detection,'' in \textit{Proceedings 2018 Network and Distributed System Security Symposium}, 2018.

\bibitem{Li2022SySeVR}
Z. Li, D. Zou, S. Xu, H. Jin, Y. Zhu, and Z. Chen, ``SySeVR: A Framework for Using Deep Learning to Detect Software Vulnerabilities,'' \textit{IEEE Transactions on Dependable and Secure Computing}, vol. 19, no. 4, pp. 2244--2258, 2022.

% ==================== TESTING FRAMEWORKS ====================
\bibitem{Antunes2011}
N. Antunes and M. Vieira, ``Enhancing Penetration Testing with Attack Signatures and Interface Monitoring for the Detection of Injection Vulnerabilities in Web Services,'' in \textit{2011 IEEE International Conference on Services Computing}, pp. 104--111, 2011.

\bibitem{Felderer2016}
M. Felderer, M. Büchler, M. Johns, A. D. Brucker, R. Breu, and A. Pretschner, ``Security Testing: A Survey,'' \textit{Advances in Computers}, vol. 101, pp. 1--51, 2016.

\bibitem{Garn2014}
B. Garn, I. Kapsalis, D. E. Simos, and S. Winkler, ``On the Applicability of Combinatorial Testing to Web Application Security Testing: A Case Study,'' in \textit{Proceedings of the 2014 Workshop on Joining AcadeMiA and Industry Contributions to Test Automation and Model-Based Testing}, pp. 16--21, 2014.

% ==================== ATTACK MODELING ====================
\bibitem{Mauw2006}
S. Mauw and M. Oostdijk, ``Foundations of Attack Trees,'' in \textit{International Conference on Information Security and Cryptology}, pp. 186--198, 2006.

\bibitem{Ficco2021}
M. Ficco, D. Granata, M. Rak, and G. Salzillo, ``Threat Modeling of Edge-Based IoT Applications,'' in \textit{Quality of Information and Communications Technology}, pp. 282--296, Springer, 2021.

\bibitem{Rak2022}
M. Rak, G. Salzillo, and D. Granata, ``ESSecA: An Automated Expert System for Threat Modelling and Penetration Testing for IoT Ecosystems,'' \textit{Computers and Electrical Engineering}, vol. 99, p. 107721, 2022.

\end{thebibliography}

\section*{Funding}
This research received no external funding.

\section*{Conflicts of Interest}
The authors declare that they have no conflicts of interest.

\section*{Data Availability Statement}
The source code, experimental data, and full documentation supporting the findings of this study are openly available on GitHub at \texttt{https://github.com/Darrii/owaspzap}. The repository includes the vulnerability chain detection system implementation, test applications scan results, chain analysis data, and scripts for reproducing the experiments.

\section*{Author Biographies}

\textbf{Dariga Yermakhankyzy} is a Master's student at the School of Information Technology and Engineering, Kazakh-British Technical University, Almaty, Kazakhstan. Her research interests include web application security, vulnerability analysis, automated security testing, and cybersecurity threat detection. She focuses on developing automated tools and methodologies for identifying complex attack patterns in web applications.

\textbf{Syed Imran Moazzam Shah} holds a PhD in Mechatronics Engineering and is an Assistant Professor at the School of Information Technology and Engineering, Kazakh-British Technical University, Almaty, Kazakhstan. His research interests include cybersecurity, software security, and secure software development practices.

\end{document}