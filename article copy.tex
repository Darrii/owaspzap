\documentclass[conference]{IEEEtran}

\begin{document}

\title{Your Research Title Here}

\author{\IEEEauthorblockN{Name Surname}
\IEEEauthorblockA{School of Information Technology and Engineering\\ Kazakh-British Technical University\\
Almaty, Kazakhstan\\
Email: s\_aman@kbtu.kz }
}

\maketitle

\IEEEpeerreviewmaketitle

% \begin{abstract}
% For the research proposal, you don't need the abstract. Uncomment this section when you will start transforming you research proposal to the research paper
% \end{abstract}

\section{Introduction}
Describe the problem. Start from general things, narrow down to your problem. 1 paragraph is enough.

\section{Literature Review}
Place your literature review here. Appx 1 page or a bit smaller. Do not forget about grouping.
\begin{itemize}
\item Export the bib from Mendeley and replace the library.bib in the project with the one you exported
\item cite papers from your bib file. If you want to add more papers, copy their BibTeX (download from the paper page) and paste to your bib file in this project.
\end{itemize}
Cite like this - \cite{Moreira2025}.

\section{Methods}
Gaps, paragraph or two.

\subsection{Data}

\subsubsection{Data Source 1}

\subsection{Proposed Approach}

\section{Results}
Describe what do you plan to do (your brick to the wall of knowledge).

\section{Conclusion}

\begin{thebibliography}{50}

% ==================== OWASP ZAP ====================
\bibitem{Potti2025}
U.-S. Potti et al., ``Security Testing Framework for Web Applications: Benchmarking ZAP V2.12.0 and V2.13.0 by OWASP as an Example,'' \textit{arXiv preprint arXiv:2501.05907}, Jan. 2025.

\bibitem{Laponina2017}
O. R. Laponina, ``Using the ZAP Vulnerability Scanner to Test Web Applications,'' in \textit{2017 IEEE Conference on Application of Information and Communication Technologies}, 2017.

\bibitem{Lathifah2022}
A. Lathifah, F. B. Amri, and A. Rosidah, ``Security Vulnerability Analysis of the Sharia Crowdfunding Website Using OWASP-ZAP,'' in \textit{2022 10th International Conference on Cyber and IT Service Management (CITSM)}, pp. 1--5, 2022.

\bibitem{Alfarizi2024}
M. Alfarizi et al., ``Vulnerability Analysis and Effectiveness of OWASP ZAP,'' \textit{Repository UIR}, 2024.

\bibitem{Jakobsson2022}
A. Jakobsson and I. Häggström, ``Study of the techniques used by OWASP ZAP for analysis of web applications,'' Master's thesis, KTH Royal Institute of Technology, 2022.

% ==================== AUTOMATED TESTING ====================
\bibitem{Bozic2019}
J. Bozic, B. Garn, D. E. Simos, and F. Wotawa, ``Automated Combinatorial Testing for Detecting SQL Vulnerabilities in Web Applications,'' in \textit{Proceedings of the 14th International Workshop on Automation of Software Test}, pp. 55--61, 2019.

\bibitem{Khan2020}
M. A. Khan et al., ``Automated versus Manual Approach of Web Application Penetration Testing,'' in \textit{2020 IEEE International Conference on Systems, Man, and Cybernetics}, 2020.

\bibitem{Feldmann2017}
L. Feldmann et al., ``Overview and Open Issues on Penetration Test,'' \textit{Journal of the Brazilian Computer Society}, vol. 23, no. 1, pp. 1--16, 2017.

\bibitem{Granata2024}
D. Granata, M. Rak, and G. Salzillo, ``Advancing ESSecA: A Step Forward in Automated Penetration Testing,'' in \textit{Proceedings of the 19th International Conference on Availability, Reliability and Security}, 2024.

\bibitem{Deng2024Pentest}
G. Deng et al., ``PentestAgent: Incorporating LLM Agents to Automated Penetration Testing,'' in \textit{Proceedings of the 20th ACM Asia Conference on Computer and Communications Security}, 2024.

\bibitem{Armando2012}
A. Armando, R. Carbone, and L. Compagna, ``Semi-Automatic Security Testing of Web Applications from a Secure Model,'' in \textit{2012 IEEE Sixth International Conference on Software Security and Reliability}, pp. 253--262, 2012.

\bibitem{Singh2018}
A. Singh and S. Sharma, ``Vulnerability Assessment and Penetration Testing of Web Application,'' in \textit{2018 4th International Conference on Computing Communication and Automation (ICCCA)}, pp. 1--6, 2018.

\bibitem{Zhang2014}
X. Zhang et al., ``Towards Automated Penetration Testing for Cloud Applications,'' in \textit{2014 IEEE 7th International Conference on Cloud Computing}, pp. 156--163, 2014.

\bibitem{Li2024DynPen}
Q. Li et al., ``DynPen: Automated Penetration Testing in Dynamic Network Scenarios Using Deep Reinforcement Learning,'' \textit{IEEE Transactions on Information Forensics and Security}, vol. 19, pp. 1--14, 2024.

\bibitem{Deng2023GPT}
G. Deng et al., ``PENTESTGPT: An LLM-Empowered Automatic Penetration Testing Tool,'' in \textit{Proceedings of the 33rd USENIX Conference on Security Symposium}, 2024.

% ==================== VULNERABILITY DETECTION ====================
\bibitem{Alaoui2022Deep}
R. L. Alaoui and E. H. Nfaoui, ``Deep Learning for Vulnerability and Attack Detection on Web Applications: A Systematic Literature Review,'' \textit{Future Internet}, vol. 14, no. 4, p. 118, 2022.

\bibitem{Chughtai2024}
M. S. Chughtai, I. Bibi, S. Karim, S. W. A. Shah, A. A. Laghari, and A. A. Khan, ``Deep Learning Trends and Future Perspectives of Web Security and Vulnerabilities,'' \textit{Journal of High Speed Networks}, 2024.

\bibitem{Tahir2024}
M. Tahir et al., ``Deep Learning and Web Applications Vulnerabilities Detection,'' \textit{International Journal of Advanced Computer Science and Applications}, vol. 15, no. 7, 2024.

\bibitem{Iqbal2015}
A. Iqbal et al., ``Web Application Security Vulnerabilities Detection Approaches: A Systematic Mapping Study,'' in \textit{2015 IEEE/ACIS 16th International Conference on Software Engineering}, pp. 1--6, 2015.

\bibitem{Singh2022Analysis}
A. Singh and A. Sharma, ``Deep Analysis of Attacks and Vulnerabilities of Web Security,'' in \textit{Advances in Data and Information Sciences}, pp. 1085--1095, Springer, 2022.

\bibitem{Kumar2017}
S. Kumar, R. Mahajan, N. Kumar, and S. K. Khatri, ``A Study on Web Application Security and Detecting Security Vulnerabilities,'' in \textit{2017 6th International Conference on Reliability, Infocom Technologies and Optimization}, pp. 451--455, 2017.

% ==================== VULNERABILITY SCANNERS ====================
\bibitem{Makino2015}
Y. Makino and V. Klyuev, ``Evaluation of Web Vulnerability Scanners,'' in \textit{2015 IEEE 8th International Conference on Intelligent Data Acquisition and Advanced Computing Systems}, vol. 1, pp. 399--402, 2015.

\bibitem{Mohaidat2024}
A. I. Mohaidat and A. Al-Helali, ``Web Vulnerability Scanning Tools: A Comprehensive Overview, Selection Guidance, and Cyber Security Recommendations,'' \textit{International Journal of Research Studies in Computer Science and Engineering}, vol. 10, no. 1, pp. 8--15, 2024.

\bibitem{Srinivasan2017}
S. M. Srinivasan and R. S. Sangwan, ``Web App Security: A Comparison and Categorization of Testing Frameworks,'' \textit{IEEE Software}, vol. 34, no. 1, pp. 99--102, 2017.

\bibitem{Alzahrani2017}
A. Alzahrani, A. Alqazzaz, Y. Zhu, H. Fu, and N. Almashfi, ``Web Application Security Tools Analysis,'' in \textit{2017 IEEE 3rd International Conference on Big Data Security on Cloud}, pp. 237--242, 2017.

\bibitem{Ghanem2023}
M. C. Ghanem and T. M. Chen, ``Enhancing Web Application Security through Automated Penetration Testing with Multiple Vulnerability Scanners,'' \textit{Computers}, vol. 12, no. 11, p. 235, 2023.

\bibitem{Zhang2020Auto}
Y. Zhang et al., ``An Automatic Vulnerability Scanner for Web Applications,'' in \textit{2020 IEEE Conference on Communications and Network Security}, 2020.

\bibitem{Touseef2019}
P. Touseef, ``Analysis of Automated Web Application Security Vulnerabilities Testing,'' in \textit{Proceedings of the 3rd International Conference on Future Networks and Distributed Systems}, 2019.

% ==================== VULNERABILITY CORRELATION ====================
\bibitem{Kasturi2024Priority}
S. Kasturi, X. Li, J. Pickard, and P. Li, ``Prioritization of Application Security Vulnerability Remediation Using Metrics, Correlation Analysis, and Threat Model,'' \textit{American Journal of Software Engineering and Applications}, vol. 12, no. 1, pp. 5--13, 2024.

\bibitem{Kasturi2023Understanding}
S. Kasturi et al., ``Understanding Statistical Correlation of Application Security Vulnerability Data from Detection and Monitoring Tools,'' in \textit{2023 IEEE International Conference on Big Data}, pp. 1--6, 2023.

\bibitem{Kasturi2024Predicting}
S. Kasturi, ``Predicting Application Security Attack Paths Using Correlation Analysis, Attack Tree, and Multi-Layer Perceptron,'' Ph.D. dissertation, Indiana State University, 2024.

% ==================== WEB SECURITY ====================
\bibitem{Vieira2024}
M. Vieira et al., ``Web Application Security through Comprehensive Vulnerability Assessment,'' \textit{Procedia Computer Science}, vol. 230, pp. 77--86, 2024.

\bibitem{Fonseca2014}
J. Fonseca, M. Vieira, and H. Madeira, ``Evaluation of Web Security Mechanisms Using Vulnerability and Attack Injection,'' \textit{IEEE Transactions on Dependable and Secure Computing}, vol. 11, no. 5, pp. 440--453, 2014.

\bibitem{Huang2017}
H. C. Huang, Z. K. Zhang, H. W. Cheng, and S. P. Shieh, ``Web Application Security: Threats, Countermeasures, and Pitfalls,'' \textit{Computer}, vol. 50, no. 6, pp. 81--85, 2017.

% ==================== OWASP ====================
\bibitem{OWASP2021}
OWASP Foundation, ``OWASP Top 10:2021,'' 2021. [Online]. Available: https://owasp.org/Top10/

\bibitem{OWASP2025}
OWASP Foundation, ``OWASP Top 10:2025,'' 2025. [Online]. Available: https://owasp.org/Top10/2025/

\bibitem{OWASPBenchmark}
OWASP Foundation, ``OWASP Benchmark Project,'' 2024. [Online]. Available: https://owasp.org/www-project-benchmark/

\bibitem{OWASPZAP}
OWASP Foundation, ``OWASP Zed Attack Proxy (ZAP),'' 2024. [Online]. Available: https://www.zaproxy.org/

% ==================== SQL & XSS ====================
\bibitem{Kieyzun2009}
A. Kieyzun, P. J. Guo, K. Jayaraman, and M. D. Ernst, ``Automatic Creation of SQL Injection and Cross-Site Scripting Attacks,'' in \textit{2009 IEEE 31st International Conference on Software Engineering}, pp. 199--209, 2009.

\bibitem{Alaoui2023XSS}
R. L. Alaoui and E. H. Nfaoui, ``Cross Site Scripting Attack Detection Approach Based on LSTM Encoder-Decoder and Word Embeddings,'' \textit{International Journal of Intelligent Systems and Applications in Engineering}, vol. 11, pp. 277--282, 2023.

% ==================== MACHINE LEARNING ====================
\bibitem{Li2018Vul}
Z. Li et al., ``VulDeePecker: A Deep Learning-Based System for Vulnerability Detection,'' in \textit{Proceedings 2018 Network and Distributed System Security Symposium}, 2018.

\bibitem{Li2022SySeVR}
Z. Li, D. Zou, S. Xu, H. Jin, Y. Zhu, and Z. Chen, ``SySeVR: A Framework for Using Deep Learning to Detect Software Vulnerabilities,'' \textit{IEEE Transactions on Dependable and Secure Computing}, vol. 19, no. 4, pp. 2244--2258, 2022.

% ==================== TESTING FRAMEWORKS ====================
\bibitem{Antunes2011}
N. Antunes and M. Vieira, ``Enhancing Penetration Testing with Attack Signatures and Interface Monitoring for the Detection of Injection Vulnerabilities in Web Services,'' in \textit{2011 IEEE International Conference on Services Computing}, pp. 104--111, 2011.

\bibitem{Felderer2016}
M. Felderer, M. Büchler, M. Johns, A. D. Brucker, R. Breu, and A. Pretschner, ``Security Testing: A Survey,'' \textit{Advances in Computers}, vol. 101, pp. 1--51, 2016.

\bibitem{Garn2014}
B. Garn, I. Kapsalis, D. E. Simos, and S. Winkler, ``On the Applicability of Combinatorial Testing to Web Application Security Testing: A Case Study,'' in \textit{Proceedings of the 2014 Workshop on Joining AcadeMiA and Industry Contributions to Test Automation and Model-Based Testing}, pp. 16--21, 2014.

% ==================== ATTACK MODELING ====================
\bibitem{Mauw2006}
S. Mauw and M. Oostdijk, ``Foundations of Attack Trees,'' in \textit{International Conference on Information Security and Cryptology}, pp. 186--198, 2006.

\bibitem{Ficco2021}
M. Ficco, D. Granata, M. Rak, and G. Salzillo, ``Threat Modeling of Edge-Based IoT Applications,'' in \textit{Quality of Information and Communications Technology}, pp. 282--296, Springer, 2021.

\bibitem{Rak2022}
M. Rak, G. Salzillo, and D. Granata, ``ESSecA: An Automated Expert System for Threat Modelling and Penetration Testing for IoT Ecosystems,'' \textit{Computers and Electrical Engineering}, vol. 99, p. 107721, 2022.

\end{thebibliography}
\end{document}